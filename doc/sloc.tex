\documentclass[10pt,letterpaper]{article}
\usepackage[utf8]{inputenc}
\usepackage{amsmath}
\usepackage{amsfonts}
\usepackage{amssymb}
\usepackage{tikz}
\usepackage{tikz-3dplot}

\author{Mosalam}
\title{SLOC}
\begin{document}

\section{Units}

The current dipole strength (of postsynaptic potentials) is \cite{matti93} 
\begin{equation}
Q = I\lambda, \;\; \lambda = (g_m r_s)^{\frac{-1}{2}}
\end{equation}
where $\lambda$ is the length constant of exponential decay, $g_m$ and $r_s$ are the conductance of the membrane and the resistance of the intracellular fluid per unit length. Hence, the unit of $Q$ is: 
\begin{equation}
Am = A \frac{1}{\frac{1}{\Omega}\frac{\Omega}{m}}.
\end{equation}


In the main equation,
\begin{equation}\overline{\sigma} \Phi( \overrightarrow{r} ) =  \sigma_{0} g(\overrightarrow{r}) + \frac{1}{4\pi}\sum_{k=1}^{L}(\sigma_{k}^{-}-\sigma_{k}^{+})\int_{s_k} \! \Phi(\overrightarrow{r'}) \frac{\overrightarrow{R}}{R^3}.\mathrm{d}\overrightarrow{S}_k(\overrightarrow{r'}) ,
\end{equation}
to verify the units of the left and right side match consider only the first term on the right hand side:
\begin{equation}
\overline{\sigma} \Phi( \overrightarrow{r} ) =  \sigma_{0} g(\overrightarrow{r}),\;\; g(\overrightarrow{r}) = \frac{1}{4\pi\sigma_{0}} \frac{\overrightarrow{p}.\overrightarrow{R}}{R^{3}}.
\end{equation}

In SI:
\begin{eqnarray}
\frac{1}{\Omega m} v & = & \frac{1}{\Omega m}  \frac{1}{\frac{1}{\Omega m}} \frac{Am\; m}{m^{3}} \nonumber \\ 
v & = & \Omega A.
\end{eqnarray}


\section{Multiplying dipole magnitude by a constant}

Let $\Phi( \overrightarrow{r}, \overrightarrow{p} ) = L_{\overrightarrow{r}} \overrightarrow{p}$ be the solution of the forward problem with dipole $p$ at location $r$. $\overrightarrow{r}_{true} = [r_{1}\;\; r_{2}\;\; r_{3}]^T$ and $\overrightarrow{r}_{true} = [r_{1}\;\; r_{2}\;\; r_{3}]^T$ are the location and the magnitude of the dipole used in the forward problem to simulate the potential measurements, $\Phi_{true} = \Phi( \overrightarrow{r}\!_{true}, \overrightarrow{p}\!_{true} )$.

You can estimate the magnitude of the dipole for the given set of true potential
measurements and the true location of the dipole by 
\begin{equation}
\overrightarrow{p^{*}}\!\!_{true} = \arg\!\min\limits_{\overrightarrow{p}} | \Phi_{true} - L_{\overrightarrow{r}\!\!_{true}} \overrightarrow{p} |.
\end{equation}

If you multiply the magnitude of the dipole by a constant scalar value, c, 
$\overrightarrow{p'} = c \overrightarrow{p}\!_{true}$, you get a new set of potential measurements, 
\begin{equation}
\Phi^{'} = \Phi( \overrightarrow{r}\!\!_{true},  \overrightarrow{p^{'}}  ) = c \Phi_{true}.
\end{equation}

Then, you can estimate the magnitude of the dipole for the given potential measurements,
\begin{eqnarray}
\overrightarrow{p^{*}} & = & \arg\!\min\limits_{\overrightarrow{p}} | \Phi^{'} - L_{\overrightarrow{r}\!\!_{true}} \overrightarrow{p} | \nonumber \\
& = & L^{+} \Phi^{'} \nonumber \\
& = & L^{+} (c \Phi_{true}) \nonumber \\
& = & c L^{+} \Phi_{true} \nonumber \\ 
& = & c  \overrightarrow{p^{*}}\!\!_{true}.
\end{eqnarray}

\section{Forward Solution}


\section{Tetrahedron}

\tdplotsetmaincoords{70}{110}
\begin{center}
\begin{tikzpicture}[scale=2,tdplot_main_coords]

	%draw a grid in the x-y plane
	\foreach \x in {-0.5,0,...,2.5}
		\foreach \y in {-0.5,0,...,2.5}
		{
			\draw[grid] (\x,-0.5) -- (\x,2.5);
			\draw[grid] (-0.5,\y) -- (2.5,\y);
		}

\coordinate (O) at (0,0,0) node[anchor=south west, color=blue]{$A$};
\draw[->] (0,0,0) -- (1.5,0,0) node[anchor=north east]{$x$};
\draw[->] (0,0,0) -- (0,1.5,0) node[anchor=north west]{$y$};
\draw[->] (0,0,0) -- (0,0,1.5) node[anchor=south]{$z$};

\draw[thick, dashed, color=blue] (1,0,0) -- (0,1,0) node[anchor=north]{$C$};
\draw[thick, dashed, color=blue] (0,1,0) -- (0,0,1) node[anchor=east]{$D$};
\draw[thick, dashed, color=blue] (0,0,1) -- (1,0,0) node[anchor=north west]{$B$};
\draw[thick, dashed, color=blue] (0,0,0) -- (1,0,0);
\draw[thick, dashed, color=blue] (0,0,0) -- (0,1,0);
\draw[thick, dashed, color=blue] (0,0,0) -- (0,0,1);

\end{tikzpicture}
\end{center}

\begin{thebibliography}{9}

\bibitem{matti93}
  H\"am\"al\"ainen, Matti and Hari, Riitta and Ilmoniemi, Risto J. and Knuutila, Jukka and Lounasmaa, Olli V.:
  \emph{Magnetoencephalography theory, instrumentation, and applications
to noninvasive studies of the working human brain}.
  American Physical Society,
  Rev. Mod. Phys, Issue 65,
  1993.

\end{thebibliography}
\end{document}